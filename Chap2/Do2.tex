\documentclass{article}
\usepackage[utf8]{inputenc}

\title{Devoir Obligatoire 2 (DO-2)}

\begin{document}

\maketitle{Cadre}

On veut simuler selon le vecteur aléatoire $(U, V)$ qui suit une distribution uniforme 
sur le domaine $\mathcal{D}_k = \lbrace (u, v) \in \mathbb{R}^2 : u^{1/k} + v^{1/k} \leq 1 \rbrace$.

\section{1ère variante: méthode du rejet}

On vous demande de coder dans une fonction simuRejection un simulateur par la méthode du rejet de cette densité. 
Pour vérifier que votre code produit le bon résultat vous pouvez graphiquer sur le plain un ensemble de points générés.
Puis, on vous demande d'étudier la dégradation du temps de calcul pour la génération de (disons) 10000 points au fur et 
mesure que $k$ augmente. Nous aurions besoin de simuler pour de valeurs de $k$ d'environ quelques dizaines.

\section{2ème variante (challenge)}

Pour simuler un vecteur $(U, V)$ on peut simuler d'abord $U$ puis selon la loi conditonnelle de $V|U$. 
Le problème est alors mathématique car il faut travailler pour obtenir les densités conditionnelles.
Vous devrez alors obtenir

la densité jointe du couple $U, V$
la densité marginale de la variable $U$
la densité conditionnelle de $V|U$
Puis coder l'algoritme d'échantillonnage (on appelle cette méthode un échantillonage de Gibbs -- Gibbs sampling)

\end{document}